\RequirePackage{plautopatch}
\documentclass[uplatex,dvipdfmx,a4paper]{jsbook}

\usepackage{amsmath}
\usepackage{amssymb}

\title{純粋関数型プログラミング言語と、数学とくに圏論}
\author{音無 恠環}
\date{2025年9月8日 - }

\begin{document}

    \maketitle

    \chapter{ラムダ計算}
	ラムダ計算は、純粋関数型言語の基盤的なモデルである。
      \section{型なしラムダ計算}
        \section{型付きラムダ計算}

    \chapter{純粋関数型言語}

    \chapter{圏論}

    \chapter{モナド}

    \chapter{Worldに対するプログラム及び数学の位置づけ}

\end{document}
